\documentclass[dvipsnames,hidelinks,t]{beamer}

  % Enables the use of colour.
  \usepackage{xcolor}
  % Syntax high-lighting for code. Requires Python's pygments.
  \usepackage{minted}
  % Enables the use of umlauts and other accents.
  \usepackage[utf8]{inputenc}
  % Diagrams.
  \usepackage{tikz}
  % Settings for captions, such as sideways captions.
  \usepackage{caption}
  % Symbols for units, like degrees and ohms.
  \usepackage{gensymb}
  % Latin modern fonts - better looking than the defaults.
  \usepackage{lmodern}
  % Allows for columns spanning multiple rows in tables.
  \usepackage{multirow}
  % Better looking tables, including nicer borders.
  \usepackage{booktabs}
  % More math symbols.
  \usepackage{amssymb}
  % More math fonts, like mathbb.
  \usepackage{amsfonts}
  % More math layouts, equation arrays, etc.
  \usepackage{amsmath}
  % More theorem environments.
  \usepackage{amsthm}
  % More column formats for tables.
  \usepackage{array}
  % Adjust the sizes of box environments.
  \usepackage{adjustbox}
  % Better looking single quotes in verbatim and minted environments.
  \usepackage{upquote}
  % Better blank space decisions.
  \usepackage{xspace}
  % Better looking tikz trees.
  \usepackage{forest}
  % URLs.
  \usepackage{hyperref}
  % For plotting.
  \usepackage{pgfplots}
  
  % Various tikz libraries.
  % For drawing mind maps.
  \usetikzlibrary{mindmap}
  % For adding shadows.
  \usetikzlibrary{shadows}
  % Extra arrows tips.
  \usetikzlibrary{arrows.meta}
  % Old arrows.
  \usetikzlibrary{arrows}
  % Automata.
  \usetikzlibrary{automata}
  % For more positioning options.
  \usetikzlibrary{positioning}
  % Creating chains of nodes on a line.
  \usetikzlibrary{chains}
  % Fitting node to contain set of coordinates.
  \usetikzlibrary{fit}
  % Extra shapes for drawing.
  \usetikzlibrary{shapes}
  % For markings on paths.
  \usetikzlibrary{decorations.markings}
  % For advanced calculations.
  \usetikzlibrary{calc}
  
  % GMIT colours.
  \definecolor{gmitblue}{RGB}{20,134,225}
  \definecolor{gmitred}{RGB}{220,20,60}
  \definecolor{gmitgrey}{RGB}{67,67,67}
  
  % Change some style options.
  \usetheme{metropolis}
  % Tell minted to use the following colour scheme. 
  \usemintedstyle{manni}
  % Remove some of the vertical space after the title.
  % \addtobeamertemplate{frametitle}{}{\vspace{-3mm}}
  % Change the default theme colours.
  \setbeamercolor{normal text}{fg=darkgray, bg=white}
  \setbeamercolor{alerted text}{fg=gmitred, bg=white}
  \setbeamercolor{example text}{fg=gmitblue, bg=white}
  \setbeamercolor{frametitle}{fg=gmitblue, bg=white}
  \setbeamercolor*{item}{fg=gmitblue}
  % Use a better math mode font.
  \usefonttheme[onlymath]{serif}
  % Don't display section pages.
  \metroset{sectionpage=none}
  % Change the default itemize bullets.
  \setbeamertemplate{itemize item}{\color{gray}--}
  % Change the position of left aligned math.
  %\setlength{\mathindent}{7mm}

  % An environment for displaying math in red, without lots of vertical space.
  \newcommand{\redmath}[1]{\vspace{-3mm} {\begin{center} \color{gmitred} $ #1 $ \end{center}} \vspace{-2mm}}

  % For displaying a blank character.
  \newcommand{\bl}{\underline{\hspace{2mm}}}

  % \citeurl can be used to a clickable short url to a slide as a reference.
  \renewcommand\footnoterule{}
  \newcommand{\citeurl}[1]{\let\thefootnote\relax\footnotetext{\tiny \textcolor{gmitgrey}{\href{http://#1}{#1}}}}
  \newcommand{\citeeg}[1]{\let\thefootnote\relax\footnotetext{\tiny \textcolor{gmitgrey}{#1}}}
  
  % Prevent minted from showing errors.
  \makeatletter
  \expandafter\def\csname PYGdefault@tok@err\endcsname{\def\PYGdefault@bc##1{{\strut ##1}}}
  \makeatother
  
  \begin{document}
    \title{Subset-sum}
    \subtitle{}
    \author{ian.mcloughlin@gmit.ie}
    \date{}
  
    \begin{frame}
      \titlepage
    \end{frame}
  
    \begin{frame}{Subset-sum problem}
  
  \begin{alertblock}{Problem}
    Given a set of integers $S$, is there a non-empty subset whose elements sum to zero?
  \end{alertblock}

  \vspace{4mm}
  
  \begin{exampleblock}{Example}
    Does $\{ 1, 3, 7, -5, -13, 2, 9, -8 \}$ have such a subset?
  \end{exampleblock}
  
  \vspace{4mm}

  \begin{alertblock}{Note}
     If somebody suggests a solution, it is very quick to check it.
    Being able to quickly verify a solution is a characteristic of NP problems.
  \end{alertblock}
\end{frame}


\begin{frame}{SUBSET-SUM}
  \redmath{ \{ \langle S,t \rangle \mid S = \{x_1,x_2,\ldots,x_k \} \mid \exists \{ y_i \} \subseteq \{ x_j \} \mid \sum_i y_i = t \}}
  \begin{description}
    \item[SUBSETSUM] is a language, as above.
    \item[Angle brackets] denote encoding their contents as a string over some alphabet.
    \item[Encoding] can be done in many ways, and Turing machines can be used to translate between different encodings, albeit with a computational cost.
    \item[$t=0$] gives a subset of SUBSETSUM.
  \end{description}
\end{frame}

\begin{frame}{Counting subsets}
  \redmath{\{a,b,c\}}
  \begin{adjustbox}{max width={\textwidth}, center}
    \begin{forest}
      [{\footnotesize \(\{\}\)}
        [{\footnotesize \(\{\}\)}
          [{\footnotesize \(\{\}\)}
            [{\footnotesize \(\{\}\)}]
            [{\footnotesize \(\{c\}\)}]
          ]
          [{\footnotesize \(\{b\}\)}
            [{\footnotesize \(\{b\}\)}]
            [{\footnotesize \(\{b,c\}\)}]
          ]
        ]
        [{\footnotesize \(\{a\}\)}
          [{\footnotesize \(\{a\}\)}
            [{\footnotesize \(\{a\}\)}]
            [{\footnotesize \(\{a,c\}\)}]
          ]
          [{\footnotesize \(\{a,b\}\)}
            [{\footnotesize \(\{a,b\}\)}]
            [{\footnotesize \(\{a,b,c\}\)}]
          ]
        ]
      ]
    \end{forest}
  \end{adjustbox}
\end{frame}

\begin{frame}{SUBSETSUM and NP}
  \begin{description}
    \item[$2^n$] is the number of subsets. Exponential.
    \item[Note] that $2^n$ is also the number of settings of $n$ Boolean variables.
    \item[Correspondence] can be seen in terms of $0$'s and $1$'s. In SUBSET-SUM the elements from the set that are included in a given subset are represented by $1$'s.
    \item[SUBSET-SUM] is NP-complete.
    \item[Usual] proof that SUBSET-SUM is NP-complete is done by reduction to 3-SAT.
  \end{description}
\end{frame} 
  \end{document}